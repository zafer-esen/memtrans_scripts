\documentclass[a4paper,12pt]{article}
\usepackage{graphicx}
\usepackage{float}

\title{Memory Bus Bit Transition Statistics for\\%s}
\date{\today}

\setlength{\parindent}{0cm}

\begin{document}
\maketitle

\section{Scalar Statistics}
\label{sec:scalar-stats}

\textbf{%s}:%s \\
\textbf{%s}:%s \\
\textbf{%s}:%s \\
\textbf{%s}:%s \\
\textbf{%s}:%s \\
\textbf{%s}:%s \\
\textbf{%s}:%s \\
\textbf{%s}:%s \\
\textbf{%s}:%s \\
\textbf{%s}:%s \\
\textbf{%s}:%s \\
\textbf{%s}:%s \\
\textbf{%s}:%s \\
\textbf{%s}:%s \\
\textbf{%s}:%s \\
\textbf{%s}:%s \\
\textbf{%s}:%s \\
\textbf{%s}:%s \\
\textbf{%s}:%s \\

\section{Graphs}
The graphs give the data either \textbf{transfer-wise} or \textbf{bus-wise}. Bus-wise refers to data sent within the same transfer (i.e. as part of the same bus transfer, as defined by bus-width). Transfer-wise refers to data from different bus transfers, which cause the bit transitions to occur.

\pagebreak
\subsection{Distribution of all byte values transmitted over the bus}

Figure~\ref{fig:dist-byte-pareto} and Figure~\ref{fig:dist-byte-pie} 
give the distribution of all 
transmitted bytes, either through a memory read or a memory write. 
The first one (pareto chart) is in terms of total number of occurrences, 
while at the same time showing the accumulation in the total percentage 
of transmitted byte values. The second one displays the absolute 
percentage of the most popular values transferred over the bus.

\begin{figure}[H]
		\centering
		\includegraphics[width=\linewidth]{%s_fig10.png}
		\caption{Distribution of all bytes transmitted over the bus - pareto chart}
		\label{fig:dist-byte-pareto}
\end{figure}	

\begin{figure}[H]
		\centering
		\includegraphics[width=\linewidth]{%s_fig9.png}
		\caption{Distribution of all bytes transmitted over the bus - pie chart}
		\label{fig:dist-byte-pie}
\end{figure}	

\pagebreak
\subsection{Distribution of transitions (transfer-wise)}
\subsubsection{All transfers (transfer-wise)}
Figure~\ref{fig:dist-tw-trans} illustrates the distribution of 
all transitions. Tuples $(n,n)$ 
have been used as $(old\_value, new\_value)$ to show the transition.

\begin{figure}[H]
		\centering
		\includegraphics[width=\linewidth]{%s_fig1.png}
		\caption{Distribution of transitioning bytes - transfer-wise}
		\label{fig:dist-tw-trans}
\end{figure}	

\pagebreak
\subsubsection{Only differing byte transfers (hamming distance $>$ 0)(transfer-wise)}
Figure \ref{fig:dist-tw-trans-diff} illustrates the distribution of 
transitions, filtered only to show differing byte transitions. Tuples $(n,n)$ 
have been used as $(old\_value, new\_value)$ to show the transition.

\begin{figure}[H]
		\centering
		\includegraphics[width=\linewidth]{%s_fig2.png}
		\caption{Distribution of transitioning bytes - 
                  transfer-wise (only differing bytes)}
		\label{fig:dist-tw-trans-diff}
\end{figure}	

\pagebreak
\subsubsection{Only same byte transfers (hamming distance $=$ 0)(transfer-wise)}
Figure \ref{fig:dist-tw-trans-same} illustrates the distribution of
transitions, filtered only to show non-changing byte transfers. Tuples $(n,n)$ 
have been used as $(old\_value, new\_value)$ to show the transition.

\begin{figure}[H]
		\centering
		\includegraphics[width=\linewidth]{%s_fig3.png}
		\caption{Distribution of transitioning bytes - 
                  transfer-wise (only same byte transitions)}
		\label{fig:dist-tw-trans-same}
\end{figure}	

\pagebreak
\subsubsection{Number of consecutive ``n'' zeros encountered (transfer-wise)}
Figure \ref{fig:consec-zeros-tw} illustrates the distribution of
consecutive ``n'' zeros encountered, transfer-wise. The x axis is
the ``n'' value, and the y axis is the frequency.

\begin{figure}[H]
		\centering
		\includegraphics[width=\linewidth]{%s_fig4.png}
		\caption{Number of consecutive \texttt{0} values encountered - 
                  transfer-wise}
		\label{fig:consec-zeros-tw}
\end{figure}






\pagebreak
\subsection{Distribution of transitions (bus-wise)}
\subsubsection{All transfers (bus-wise)}
Figure~\ref{fig:dist-bw-trans} illustrates the distribution of all 
transitions happened during a whole benchmarking, bus-wise. Tuples $(n,n)$ 
have been used as $(old\_value, new\_value)$ to show the transition.

\begin{figure}[H]
		\centering
		\includegraphics[width=\linewidth]{%s_fig5.png}
		\caption{Distribution of transitioning bytes - bus-wise}
		\label{fig:dist-bw-trans}
\end{figure}	

\pagebreak
\subsubsection{Only differing byte transfers (hamming distance $>$ 0)(bus-wise)}
Figure \ref{fig:dist-bw-trans-diff} illustrates the distribution of 
transitions happened during the whole benchmarking. Tuples $(n,n)$ 
have been used as $(old\_value, new\_value)$ to show the transition.

\begin{figure}[H]
		\centering
		\includegraphics[width=\linewidth]{%s_fig6.png}
		\caption{Distribution of transitioning bytes - bus-wise (only differing bytes)}
		\label{fig:dist-bw-trans-diff}
\end{figure}	

\pagebreak
\subsubsection{Only same byte transfers (hamming distance $=$ 0)(bus-wise)}
Figure \ref{fig:dist-bw-trans-same} illustrates the distribution of
transitions, filtered only to show non-changing byte transfers. Tuples $(n,n)$ 
have been used as $(old\_value, new\_value)$ to show the transition. 
\begin{figure}[H]
		\centering
		\includegraphics[width=\linewidth]{%s_fig7.png}
		\caption{Distribution of transitioning bytes - bus-wise (only same byte transitions)}
		\label{fig:dist-bw-trans-same}
\end{figure}	

\pagebreak
\subsubsection{Number of consecutive ``n'' zeros encountered (bus-wise)}
Figure \ref{fig:consec-zeros-tw} illustrates the distribution of
consecutive ``n'' zeros encountered, transfer-wise. The x axis is
the ``n'' value, and the y axis is the frequency.

\begin{figure}[H]
		\centering
		\includegraphics[width=\linewidth]{%s_fig8.png}
		\caption{Number of consecutive \texttt{0} values encountered - 
                  bus-wise}
		\label{fig:consec-zeros-bw}
\end{figure}

\pagebreak
\subsection{Byte reuse ratios at eviction}

Figure~\ref{fig:most-reused} and Figure~\ref{fig:least-reused} 
give the reuse ratios of the bytes for each byte value. Note that the
byte values are the last values the cacheline contained \textbf{at eviction},
not the original byte values brought in to the cache. Also note that these 
numbers do not give an idea on \textit{how much} the cacheline was reused. The
values in the cacheline are marked as reused as soon as the value is read/written to
after the initial access.

For the overall cacheline reuse statistic, see Section~\ref{sec:scalar-stats}.

\begin{figure}[H]
		\centering
		\includegraphics[width=\linewidth]{%s_fig11.png}
		\caption{Most reused byte values at eviction}
		\label{fig:most-reused}
\end{figure}	

\begin{figure}[H]
		\centering
		\includegraphics[width=\linewidth]{%s_fig12.png}
		\caption{Least reused byte values at eviction}
		\label{fig:least-reused}
\end{figure}

\end{document}
